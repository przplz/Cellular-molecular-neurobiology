\documentclass[a4paper, 12pt]{book}
\usepackage[english]{babel}
\usepackage[T1]{fontenc}
\usepackage[utf8]{inputenc}
\usepackage{todonotes}
\usepackage{caption}
\usepackage{subfig}
\usepackage{textcomp}
\usepackage{float}
\captionsetup{format=hang,labelfont={sf,bf}}
\usepackage{amsmath, amssymb}
\usepackage{xspace}
\usepackage[hidelinks]{hyperref}
\usepackage{mhchem}


\newcommand{\freccia}{\ensuremath{\rightarrow}}
\newcommand{\lfreccia}{\ensuremath{\longrightarrow}}
\newcommand{\gradi}{\textcelsius\xspace}
\newcommand{\ang}{\AA \xspace}



\begin{document}
\author{Elisa Nerli}
\title{Cellular and Molecular Neurobiology}
\maketitle
\newpage
\tableofcontents
\newpage

\chapter{Lesson 1: Cellular organization of the NS}
Dr. Enrico Tongiorgi.
Morphology and function aspects of neurons and molecular details. 

Exam:  Written + oral: true-false questionary. The questions are subdivided lesson by lesson. Oral exam is optional, focused on the lessons which I failed. 
 Book: Alberts, Fundamental Neuroscience.
\\

In the CNS (spinal chord, brain and retina):
\begin{itemize}
\item{Gray matter: nerve cell bodies.}

\item{White matter: axons.}

\item{Neuropil: areas in which axons and dendrites synapse.}

\item{Nuclei: regions in the brain which contain a collection of nerve cell bodies. If they go left to right \freccia fasciculi, peduncule. The contrary: lemnisci, tracts.}
\end{itemize}
In the PNS (sensory and autonomic ganglia) :
\begin{itemize}
\item{Ganglia: collection of nerve cell bodies.}
\item{Nerves: collection of axons.} 
\end{itemize}

In the human brain we find 1 billion of million of cells, that are neuron and glial cells.
 These cells express the largest number of protein in the human body.

\section{History of neuroscience}
Aristotel believed that the mind was in the heart, because heartbeat changes following emotions. Hyppocrates considered that it was in the brain. In particular, he was proposing the theory of four fluids or \emph{humours}:
\begin{enumerate}
\item{Sanguis or blood, produced by the heart}
\item{Choler or yellow bile, produced by the liver}
\item{Melancholia or black bile, produced by the
spleen}
\item{Phlegma or phlegm, produced by the brain}
\end{enumerate}
Later, Galen tried to explain the functions of the brain elaborating the dottrine of fluids. In Baghdad there was a school for the development of psychology and physiology: there was the first mental hospital in the world. They were divided the mental illness into two part: the medical part and the psychological part\footnote{Musical therapy was already in use at that time}. \emph{Ibn Sina} is the father of Neuroscience, in particular of the cognitive functions (logic, memory etc). He was the first to question the Galen theory of fluids. In 1867 they started to understand that brain contains different functions.

1882: all body tissues are formed by cells (cell theory).  The brain was still a mystery!

1906: thanks to the possibility to fix tissues, Camillo Golgi found out the \emph{Black reaction}, a method in which golden and silver were used (fist nobel in neuroscience).

1950: selective silver staining.

1970: antibody staining, fluorescent markers.

\paragraph{Golgi staining}
Potassium chromate and silver nitrate. 
Nerve cells, instead of working individually, act as a network. He was saying that neurons form a continuous network, but that is not correct. Anastomotic means sincitiuum (sincizio). Golgi believed that nervous cells are a sincitiuum.

\subsection{The Neuron doctrine from Santiago Ramon y Cajal}
Neurons are cells. Each neuron is an individual entity anatomically, embriologically and functionally. They form a network, but they are separated to each other. Neurons have also a polarity (visual observation, but he couldn't demonstrate it). Recurrence of similar structures.

Structure of neuron: \todo{put picture}
Spines: specializations of dendrites. 

The morphological polarization underlies the functional polarization. There are some peculiarities, some recurrent aspects:
\begin{itemize}
\item{All neurons have dendrites, soma and axon}
\item{They might have different shapes}
\end{itemize}
So they observed that there are \emph{sensory neurons} (CNS, except for the cell body), \emph{interneurons} (90\% of the neurons in the brain) and \emph{motor neurons}. 

\subsection{Structure of a neuron}
Cell body = soma. It may be smaller or larger and this helps classifying neurons. Single, central and prominent nucleolus (the site of production of proteins, meaning that the cell is producing a lot of proteins). The cytoskeleton is very ramified to keep the shape. We need RER (rough endoplasmic reticulum), ribosomes and Golgi apparatus (a lot!). We can see tiger dots (isole tigroidi). 

Dendrites: from 1 to 10, shorter than the axon, branched.

Axon: just 1, it starts from the axon hillock (small hill, colletto) and ends with a ramification (synapses). It may have the myelin. Excitatory synapses are made between axon and dendrites, inhibitory with the soma.

\subsubsection{Comparison with somatic cells}
Not so many cells have these strange, long ramifications (over a meter in some cases). In somatic cells, like epithelial cells of the gut, there is an apical membrane and a basolateral membrane. There are proteins and receptors that are exclusive for the apical part and others for the basolateral part. Is there also this membrane diversification in the neuron, or there is a mix in terms of proteins? They used viruses that produce different proteins: injected, in the gut some proteins were going up and some down. In neuron cells, some were going in the axon and some in the cell bodies \lfreccia membrane polarization also in neurons. That is non true in C. Elegans, in which these proteins are mixed.

Microtubules orientation determines the polarity of neurons.

Epithelial cells have a polarity because the apical surface looks into a lumen. The lateral surface has tight junctions. In neuron, somatic and axon membrane are separated by the axon hillock. The axon hillock is a molecularly distinct domain: it contains protein and channels that are important to starting the action potential.


Anterior Horn Motor Neuron (corna anteriori). Nissl stains (it's a colorant) colors areas which are rich in RNA.

The similar structures that Cajal was talking about, neurons, can be classified in:
\begin{itemize}
\item{Multipolar neuron: most common, multiple dendrites and one axon}
\item{Bipolar: one dendrite/one axon. Common in retina, ear, olfactory (is the typical sensory neuron}
\item{Unipolar neuron: typical sensory neuron: out of the cell body one single process comes out, a half has the function of dendrites (but is not a dendrites) and one works as an axon. Morphologically are both axons, but not functionally (pseudounipolar). Cell bodies are collected in the ganglia \lfreccia PNS}
\item{Anaxonic neuron: only in the retina}
\end{itemize}

The mammalian brain contains many multipolar neuron, while in invertebrate there are may unipolar neurons. Morphology is strictly linked to function. In mammalian brain we have to integrate lots of informations \lfreccia that's why we have lots of multipolar neurons, to \emph{integrate informations}. Fish or invertebrate have very simple reflexes, they don't have to integrate many information. Neuronal somata are collected in rind or ganglia in invertebrates, while in vertebrate are grouped in nuclei in the CNS. Invertebrate lack myeline, because the axons are short and because they have simple reflexes and neurons are smaller, bigger and unipolar: they compensate the lack of speed that vertebrate would have if their neurons lack myelin\todo{Summer school in SISSA sul sistema nervoso degli invertebrati!!!!!}.

Multipolar neurons are very different: the classification of Santiago Ramon y Cajal is too simple. In the cerebral cortex, there are different layers that host different kind of neurons.

In the Hippocampus, we find \emph{pyramidal neuron}.  

Dendritic Spines are 10-40000 each neuron. In retina, there are lots of type of cells, like \emph{Starbust}.
For the classification, the largest difference is in the dendrites.

Morphometric analysis: analysis of dimensions, shapes etc changes in dendrites. 

\subsubsection{Characteristic arborization patterns}
So far we have considered neurons as flat, but they are 3D structures. The newest classifications comprehends this observation. The stellate neurons can be flat or occupy a sphere like a sea urchin or a half-spherical distribution. It depends on what kind of information you want to collect. We can have also a laminar radiation: cell body can be inside or outside; neurons can occupy cylindrical structures, biconical radiation or fan radiation, uniq for the pyramidal cells. 

The functional meaning is: bipolar neurons collect informations only by few neuron. If ramifications are a lot, like in Purkinje fibers, the neuron collect information from every neuron in that area. 


\chapter{Lesson 2: Neuroglia}

6 types: 
\begin{itemize}
\item{Astrocytes}
\item{ Microglia}
 \item{Ependimal cells}
 \item{Radial glia}
 \item{Oligodendrocytes}
 \item{Schwann cells} 
\end{itemize}
They are protecting the NS because they sequestrate the ions in excess, they give metabolic support.  Astrocytes have contact with neurons and capillaries. 

\section{Functions}
Astrocytes have a physical support, trophic, signaling, homeostasis function. They are also star-shaped and symmetrical.

Microglia is the resident immune system: these are small cells, mesodermally derived. Microglia is very small and difficult to detect.

Oligodendrocytes are the cells that create myelin in CNS, they can collect many neurons; Schwann cells are in the PNS and  can collect just 1 neuron.

A typical features of astrocytes is the stellar shape with spines and enlargement called feet. The processes contact the vessels.

 \section{Gliogenesis}
The making of the glia. It starts from the \emph{neuroepithelial stem cells}\footnote{pluripotent cell}, that will generate neurons or glial cells. Thanks to the presence of specific factors, PDGF and FGF2, there is a commitment of these cells, which will become \emph{multipotent} cells. They are now committed to generate a glial precursor. After further interaction with others growth factors, there are 2 pathways that lead to the formation of the precursors of glial cells:
\begin{itemize}
\item{The type 2  astrocyte (from the pathway with PDGF and CNTF)  and oligodendtrocytes (with inly CNTF) are the final stage of maturation}
\item{The type-1 astrocyte (from the pathway with CNTF and EGF) has no processes and can generate the adult neural stem cells}
\end{itemize}

\subsection{Microglia}
They behave like macrophages. Their origin is not from neuroepithelial stem cells. CNS is not completely isolated from immune reactions. They can be marked with lots of molecules. They can contribute to several disease, in particular those who concern inflammation in the brain.

Microglia is important in most pathologies in the CNS, because its activation is an attempt of the brain to react to possible damages. It can eat red blood cells after a stroke, because red blood cells contains ions.

The microgliogenesis  occurs in the \emph{bone marrow} (midollo osseo),  from the monocyte line: during the development they can pass through the blood vessels inside the brain\footnote{Those who migrate outside becomes macrophage}, where they find their position and differentiate to a typical microglia. They need to have an ameboid behavior. During the development, the cells continue to migrate to expand even in the new area, and they also proliferate (but they need to stop at a certain moment). The activation depend on the secretion of ATP from the neurons and the M-CSF and GM-CSF from the astrocytes (they are also present in the rest of the body). Also many toxic factors can activate the microglial cells, and also some of the neurotrophic molecules.

If you eliminate glial cells from the brain with irradiation of the brain during development, death is the result.

\subsection{Ependimal cells}
Epithelial-like structure, cube cells. They are located in the outer surface of the brain, under the meningi, and in the spinal cord canal They are also present on the surface of the choroid plexus\footnote{cells that produce the liquor}.  They are ciliated, with adherence junctions, and express the glial markers. They have the intermediate filaments; they can extend cytoplasmic processes into the brain parenchyma. NCS derives from astrocytes.

\subsection{Radial glia}
It has some radial structures that go from the lumen of the spinal cord (the neural tube in embryogenesis) to the surface of the NS during the embryogenesis. In the central zone of the lumen the newborn neurons are created from the neuroepithelial precursor cells: the neuroblasts migrate along the radial glia. The radial glia produces a matrix and adhesion proteins to help the neuroblasts moving. 

The radial glia is not only present during the development, but persist in the cerebellum, where is called \emph{Berman glia}, and in the retina, where is called \emph{Muller cells} and is found with a marker...

\subsection{Astrocytes}
They are called star cells, the most numerous cell type in the brain. Important for the migration of neurons during development, for the blood-brain barrier (induction and mantainance). 
There are 2 types of astrocytes:
\begin{itemize}
\item{Protoplasmic: pedunculated, are typically located in they gray matter. They are in contact with blood vessels}
\item{Fibrous: in the white matter}
\end{itemize}

They produces neurotrophic factors, they buffer excess of ions and transmitter in the extracellular space, endocyting it. Then, they car or destroy these substances for neuron to recycling, or they can ...?
They protect neurons from the excitoxicity, that can cause neuron degeneration. They release specific neurotransmitters only secrete by glial cells.

Astrocytes can contact basically \emph{every} cell component in the brain! They can control every function of the brain. Just like the microglia, they can divide when stimulated: dangerous if a cell come in a tumoral behaviour. There are probabilly different pattern of activation of these cells.

\section{Blood-brain barrier}
The end feet of the astrocytes form the cover of the blood vessels. The barrier has been discovered in 1885. The continuous basement membrane coats the vessels.This barrier is also present close to the ependimal cells, on the surface of the brain: it is flat and cover perfectly the brain, and it is called \emph{the glia limitans}.

What is the blood-brain barrier?

A structural and functional barrier that prevents the entrance of molecules in the brain, selectively. Formed by BMEC (brain microvascular endothelial cells), astrocyte end feet and pericytes, that are surmounted by the astrocytes and circondano the vessel. It is essential for normal function of CNS because it manteins the neural environment in and out of the brain. It is not present in the ipothalamus. It regulates the concentration of L-dopa.
The nucleus of astrocytes is far away from the vessel. In somatic capillary there are fenestra, not in the NS. There is also a neuron in the brain capillary on the top of the endothelium: it helps to regulate blood stream pressure because the capillary can expand o restrict following the secretion of neurotransmitter from the neuron. 

Tight junctions are present only in this barrier, the molecules are claudin and occludin, that are in contact with the cytoskeleton. We find also junctional adhesion molecules.

The cerebral vasculature can change volume according to the need of $O_2$. In absence of oxygen, neurons can survive 3-6 minutes. The vasculature has an autoregulation, but also can be regulated by the neuronal activity.

A molecule that has to pass the barrier, cannot go through the plasma membrane or between the cells, but it must go on a vesicular transport (receptor mediated endocytosis or pinocytosis). The integrity of the barrier relies on the presence of tight junciont, adherens junction, pericytes and atrocytes.

\subsection{Tight junctions}
Made by claudin and occludin and junctional adhesion molecules (JAM). The occludin are regulatory proteins because they can alter the paracellular (intercellular, or between the cells) permeability. Occludin and claudin assemble into heteropolymers and form ... The loss of occludine is present in breakdown of BBB.

JAM are integral membrane proteins, they are also involved in the monocyte transmigration over the BBB and in the regulation of paracellular permeability. They may have a homotypic binding (interaction with homologous molecule) and a heterotypic binding with leukocyte, like in multiple sclerosis. 

There are also accessory proteins, like ZO-1, ZO-2, ZO-3, cingulin, that link the occludins  to the cytoskeleton in order to make a tight junction very solid: they are globular proteins. Also MAGUKS, guanylate kinases, make the same thing, but can be regulates and they can attach or detach.

Caderins are adhesion molecules present in every tissue.

\subsection{Pericytes}
Cells that provide structural support and are important for the phagocytic activity: they are the first calls able to destroy some damaging molecules.

\paragraph{}

There are regions that are not included in the BBB:
\begin{itemize}
\item{Circumventricular organs: they secrete hormones in the blood stream, like neurohypophysis, pineal gland, lamina terminalis}
\end{itemize}

With facilitated transport can be carried amminoacids, glucose (a lot!), glycine.  
Some fat soluble molecules can cross the BBB and cell membranes, like some drugs or treatments.
A lot of factors can alterate the BBB and provoke disease or alterate the mood.

\chapter{Lesson 3: Neuroglia and myelin}
\section{Oligodendrocites}
Oligodendrocites are few branch glia in the CNS. They have been discovered with the technique of Golgi and Ramon y CAjal. The myelin contituets the white matter: they are target of some disease like multiple sclerosys (may occur in young people, is an immune attack against oligo).

Olidodendrocites form myelin around more than one axon. One of the typical feature is that these enrollement of the axons are formend only on discrete portion an there is an interaction of two portion of myelin that is the Ranvier's node. Myelin is formed by a series of levels which has elettrondense and elettronclear appearence at the microscope. 

\section{Schwann cells}
Schwann cell is the second type of myel forming glial cells, typical of the PNS. The appearence of the myelin coating the axons is the same of oligo. The main difference between oligo and schwann is that one schwann cell myelinates one axon, an oligo myelinates more than 1 axon.

A pseudounipolar neuron has 2 axons myelinated, but one acts like a dendrite, because the direction of the conduction is from the perifery to the soma. In the CNS, the myelin is exclusively in the axons, in PNS there is also on a axon working as a dendrite (in pseudounipolar neuron). In a motor neuron, soma in in the CNS and axon in the PNS to innervate muscle.

Not all the axons are myelinated in PNS: they are involved in pain detection, and that's the reason why we sometimes feel pain after a certain time. That's because the signal conduction in these fibers is much slower. 
\subsection{Nerves}
Once the myelinated sensory fibers and the myelinated motor fibers are out of the CNS, they are pached into \emph{periferal nerves}, and there is a mix of sensory and motor fibers, and they are surrounded by a matrix composing a connective tissue, composed of fybroblasts, fibers like fibronectin, laminin, together with other substances that help keep n position the myelinated fibers, protecting them. There are also elastic fibers to have flexible nerves, to stretch them. In black we see the \emph{periniuron} (inside there is the \emph{endonireon}, that contains lots of water to help the passage of substances), more fibrous. The outside contains some fat cells and is the \emph{epiniuron}, the connective tissue that covers the entire nerve. The nerves can be colorated with ematoxilin or...

The capillaries blood vessels are imbended into the outer fibrous coat, the epineuron. 

The periniuron is composed by flat epithelium, lots of connective fibers. The endoniurion we have a structure containing the Renault bodies, that are internal sites for immune system to protect the nervo from any viral or bacterial attacks: the virus can enter into the cytoplasm of these cells, crossing the myelin, and reach the cell bodies into the spinal cord, for exemple after an accident that leads to a break of the nerve.
Large capillaries are in the epineuron, bud small capillaries are also inside the endoniurion to bring glucose and oxygen: in the endoniurion there are fybroblasts that produce connective tissues, there are some resident immune cells like macrophages, capillaries and the unmyelinated axons as well as myelinated fibers.

Unmyelinated fibers are the invaginations in the glial cell: the diameter of the myelinated axons is larger that the unmyelinated. The glial cells so protect the axons and has a trophic function and exchanging signals.

\section{Myelin}
The myeli act as insulator for the \emph{vertebrate} nerve cells, formed by wrapping of plasmalemma around the axons. 80\% lipids and 20\% protein. All myelination is completed at 25 years old, and lady have a faster myelinization. We have a pagkaging of this plasma membrane one on top of the other, and the cytoplasm is eliminated by a physical squeezing. 

What influence the speed of a nerve signal is the diameter of the fibers and the presence of the myelin: larger fibers have a bigger surface for signals and:
\begin{itemize}
\item{Small, umyelinated fibers: 0.5-2.0 m/sec}
\item{Small, myelinated fibers: 3-15 m/sec}
\item{Large, myelinated fibers: up to 120 m/sec (almost istantaneous}
\end{itemize}

Slow signals supply the stomach and dilate pupil.

In myelinated axons we have a saltatory conduction between the Ranvier's nodes. In order to achieve these, in the nodes there are some specializations: the channels that open and allow ions to pass across the membrane are Na channels, at very high density.

The composition of the myelin: the major dense line (very dark at the microscope) is the cytoplasm; the withe part is the plasma membrane and the red is the extracellular part. . The wrapping of the plasma membrane are crossed by some transmembrane proteins: there are \emph{cell-cell adhesion}, so these proteins are cell adhesion molecules that compact the myelin to have a very dense packaging and squeeze out the cytoplasm. There is the myelin basic protein, MBP, that helps the compactation and the squeeze of cytoplasm.

There are some tumorsial cells of gl: neurofibroma and schwannoma in PNS, atrocitoma (benign pilocytic astrocytic and the most common is malignant, the glioblastoma), oligodendrogliomas and ependymomas (is the more easily removed by surgical interventi) in CNS. Some diseases are due to mutations that do not permit a good compact of the myelin, like because there is a mutation in MBP. 

The diameter of the myeline is different during the age process. There are 2 big cluster of myelinated diameters, one around 5 um and one aroun 12 um

\subsection{Mielination}

In myelination the cells need a continuous contact with the axons, so the wrapping in conducted to the contrary of what expected (il primo avvolgimento sara il piu esterno, il secondo sara subito sotto ecc, in questo modo la cellula mantiene il contatto con l'assone sempre!).

The contibution of the external part that is inconnection with the cell body in limited: only the free edge can activley produce the myelination. The cell recognized that the first round is completed, because there are some signals in the contact point, and continues growing. There are spaces between the axon and the leading edge, because the leading edge still has cytoplasm, because it is needed to produce the cytoskeletal compound needed to the growth. The leading edge is growing and behinds the proteins are compacting the wrapping and the cytoplasm is squeezing. How can the cell mantein the contact with the edge? There are small channels, narrow corridor in which the membranes are not compacted (and it is the white part). The end point of the myelin needs to be closed to evit the passage of everithing, but there is also an enlargment with cytoplasm for the communication. 

During myelination, how can the cell decide when is the moment to compact? The proteins of adhesion are produced from RNA granules: in order to make sure that myelination occurs only when needed, the rybosomes that are packed into the RNA granules receive a signal, coming from the contact between the axons and myelin, that open the granules and the translation of the adhesion molecule stars, for compacting of the myelin. 

The position of the myelin on the axon causes a movement of the surface molecules of the axon: channels move laterally, they are eliminated from the places that will host the myelin and go into the Nodes of ranvier, also because initially we have several adhesion molecules dinstibuted on the surface of the Schwann cell (in blue): these are proteins which are called \emph{neurofascin 155}, associated with the cytosckeleton. They takes contact with the contactine, anther adhesion molecule on the axon, and then these move at the top of the myelin wrapping during the compaction, and they remain in the part in which we have enlargements. Here they provide signals for the cytoskeleton and the Na channels are moved laterally, because the adhesion molecules are linked to the cytoskeleton. The two neighboring Schwann cells position the channels between them, in the Node. The area with the neurofascin is the \emph{paranode}: this displace a molecule that is the ankirin that take the Na channels in the middle of the nodes of ranvier.

NF 155 are present on the membrane of the glial cell and take contact with a receptor on teh surface of the axon that is attached to a cytoskeletal. The molecule move laretally into the paranodal place, and moving they push the Na channel into the nodes of Ranvier, because the Na Channels are anchored to ankirin. Ankirin is bring into position from the paranode region: now gap and tight junctions close the terminal part (between the different wrappings of the glial cells).

Ankyrin binds to specrtin and they are bound to the channels.


\chapter{Lesson 4: Astrocytes functions}
(15-10-2015)

Historically recognized:
\begin{itemize}
\item{Sequestration of K+ during neural activity. K+ are concentrated outside and, if there is too much K+, the astrocytes will remove it}
\item{Removal of neurotransmitter and recycling of them (GABA and glutamate) to make precursors}
\item{BBB regulation}
\item{Involved in neural stem cells generation (regulation of neurogenesis)}
\item{Regulation of synaptogenesis (establishment of new synapses}
\item{Modulation of synaptic activity}
\end{itemize}

Presence of astrocytes closely related with synapse, in particular to the synaptic cleft: over 90\% of synapses have astrocytes. Take a brain embryo from rodents, dissociate cells using trypsime and plate them in a Petri dish in appropriate medium: if we take at day 12 of gestation, there are no glial cells that are formed \lfreccia 90-95\% of the cells in the plate will be only neurons. Synaptotagmin is present in the presinaptic, PSD95 in the post synaptic, these are 2 proteins: if we see them in the same place, that means there is a synapse, because they are co-localized. In a culture using the same neurons, but on the top also a culture of glial cells, we can see both proteins, while in the culture of only neuron we cannot \lfreccia the presence of glial cell plays a role in synapses formation.

The number of the glial cells grows growing the complexity of the animals. Albert Einstein has much more glial cells per neuron than a normal person. 
\paragraph{Experiment 1}
Astrocytes use Ca to "talk", meaning that the excitability of astrocytes is caused by a variation in intracellular Ca concentration: taking a retina and put this in culture. The retina can be filled with a fluorescent dye which response to Ca increase: every time the fluorescence increase, there is a Ca increase. Touching the cells provoke a flash of fluorescence: this explains why when we get a hit in the eye, we see flashing of light. When we touch the cells with a glass pipette e see a circular wave that goes from one cell to another. There are specific connections between astrocytes that are GAP junctions, that form membrane pores: the Ca ions can freely pass from one cell to another \lfreccia wave (short-range signals). These waves stops after certain distance; there are also long-range signals that go through membrane receptor, because glial cells and neurons can release some specific \emph{gliotransmitters}\footnote{ATP, Glu, D-Ser, which regulate neuronal excitability and synaptic transmission}. In particular, ATP\footnote{given by neurons and glia} has a receptor on the top of glial cells, connected to PLC, that activates the IP3, that activate the RE \lfreccia release of Ca from the RE. These two systems work in parallel, they are both activated.
\paragraph{Experiment 2}
Another experiment: it is possible to use mouse models that were grafted with astrocytes coming from human, so they have much more contacts\todo{leggi l'articolo} . The chimeric learned very well to respond to stimulus. 
	\paragraph{Experiment 3}
Another experiment: astrocytes loaded with IP3 caging: with a flesh of light we can uncage the compound \lfreccia increase in the astrocytic content of Ca. At the same time, they put an electrode in the neighboring neurons and recover a spontaneous potential. The average frequency of this spontaneous activity in increasing: this activity is caused by a release of neurotransmitters vesicles. The signal that make this release is: by rising the intracellular Ca in astrocyte, there is a possibility to release Glu, because it is captured by astrocytes and release again, that will activated the presynaptic metabotropic receptor, ad D-Serine, that can activate a part of the Glu receptor, the Gly site (this site can bind D-Ser and Gly): this binding let the receptor opens better. Gly is released by neurons and D-Ser form astrocytes: in this way the astrocytes can modulate the release of Glu by the neuron.

It is possible to measure the electrical activity of the neuron: they found out that astrocytes are activated with the neuron.
\\

Let consider a synapsis which release ACh. This synapse is wrapped by glial cells. Making the experiment 1 and comparing the synaptic activity, they found out that when astrocytes were presents, there wasn't a better transmission! O.O There was a decrease in synaptic activity and the ACh was captured by a molecule produced by glial cells, capturing and eliminating ACh.  But how the glial cells sense and respond to the ACh? They have a receptor for ACh that is active and responds to an increase of ACh: the response is the production of ACh binding protein that is release in the synaptic cleft. This protein has been identified by Bungarotoxin purification scheme. This protein, given alone without the glial cells, produces the same synaptic depression. 

All this plasticity is accompained with morphological changes. The change in the volume over time\footnote{Changing the concentration of the neurotransmitter ins the synaptic cleft} is more in the processes that in the spines of astrocytes. So, astrocyte can modulate the concentration of the neurotransmitter just increasing their volume into the synaptic cleft. This aspect of the volume is important also for the function of the glia: if glia is very closer to the cleft, there is a higher probability to catch the neurotransmitter.


How can we relate the changing shape and all the molecules described?

The actin cytoskeleton is modulate by Ca: ATP and Glu receptors on the surface of the astrocyte and a pool of actin single molecule that can be added to a growing actin filament. To push the membrane we need something: this is the cytoskeleton. We need to build the cytoskeleton with actin filaments, that are parallel and grow into a direction, and this is controlled by Ca and phosphate groups. Profilin is a proteins that helps in building the actin filaments. Overexpressing this protein we can change the shape of the astrocytes.

Important: the anesthesia takes down the Ca frequency. There is also a difference between young and adult brain in the function of astrocytes: young astrocytes are more reactive to synaptic activity.


\chapter{Lesson 5: Dendritic spines}
Pyramidal muktipolar neruon: apical dendrites and several smaller basal dendrites. They have some small protrusion of the surface and show the highest density of number of spines (number diviso length). 

Most of the studies are done with neurons filled with fluorescence proteins. Spines are very different but there are specific features: SER, spine apparatus (cellular cisterne), polyribosomes\footnote{They can be also in the periphery of the neuron}. Spines are twitching: in addition to the classification of the shape, we have to deal with the motility of these spines. In particular, in learning  and memory spines contributes substantially in changing their shape. Treating spines with isofluorane is going to immobilize the spines.

The shape is important because a loss of the shape is linked to certain pathologies: if there is a neuronal developmental disorder, like mental retardation, the shape of spines is involved and a decrease in number of spines.

Spine density decrease normally during normal aging and a change in shape during hormonal change. Glu and growth factors are important to modulate the nubmer and shape of spines. There are fillopodia, the first structures that take contacts with the axon, in order to establish synaptic contact. There is a polimerization of actin filaments and the fillopodia is produced: then it establish the contact, that can be temporary or not, establishing a synapse.

Once created, spines will not rest like that forever, but they will move ant they require a continuous signaling to be maintained. Spines that are not required will be selectively\footnote{What is not stimulated is lost} eliminated: this process  is called \emph{pruning} and occurs during the development a phase in which there are a lot of spines. At 11 years old we have the maximum number of spines, then there is a decline during the teen-age time.
Fillopodia come out rapidly and are rapidly retracted. 

In the early stage of development (10 days in vitro) we can se several fillopodia, while in the mature stage there are more established synapses.

\section{Dendritic spines types}
By looking at the Golgi staining, there is a classical classification that distinguish 3 types:
\begin{itemize}
\item{Stubby}
\item{Thin}
\item{Mushroom}
\end{itemize}

The shape is mutant in time. There is a \emph{mathematical rule}: some spines (type 1) have a diameter of the head is the same of the length. Type 2 there were longer than large, ad had a thin neck; type 3 have a very long neck.
Type one are stubby spines;  type 2 are mushroom spines; type 3 are thin spines. 

How we distinguish a filopodium from a thin spines? by the motility, that is more rapid than a spine, and the length: a spine is smaller that 3 $\mu m$ \lfreccia there is also a maximal length of the spines. 

The mushroom spines can also have a cup shape. Sometimes we can se a spinule covered by clatrine.

In the thin spines, the connection with the dendrite is very narrow, while the stubby shape have a complete connection with the dendrite. In red excitatory synapses, in blu the inhibitory: note that the red synapses are placed on spines and the blue one on hte parental dendrite.

In the pedunculated spines: large head and small neck. . In the simple spine: no constriction between the spine and the dendrite, so the exchange of material is easy. This means that there is not a restriction in exchange with synapses! The morphological shape and its function are related. We can have also thin spines that have a small neck: there are lots of possible variations!

In a dendrite we can find en enlargement, the \emph{varicosity}, very common in invertebrates. The synaptic crest is very extendend in 3D, even if it looks like a mushroom.
The aim of complexes spines is to have contact with different axons to make an integration of different signals. In the case of thorny excrescence, it is one of the most important structure for the memory: they are so big because they are the most active synapses in the brain and is important to integrate different memories and recall sequential memory.  During the evolution, the hippocsmpus was the site of the spatial memory, but now we know that also the time memory is built it this site. These synapses are so strong that cells die and have to be renovate. 

Several of these complex spines are present in the cerebellum: they also make spines biforcated to maximize the number of contact.

\subsection{Internal structure}
Basically, the most highly concentrated molecules are related to the cytoskeleton and are responding to Ca: there are a lot of actin filaments and in the central part of the spine we find filaments that are called \emph{stable actin}: they don't undergo an intensive polymerization od depolymerization because they have a cap on their endings (yellow and bleu points). To keep a stable actin filaments, we have also to distribute the actin filaments in a certain direction \lfreccia we need to bind them with transversal proteins, the $\alpha$-actinin. We have seen that spines are twitching: the most mobile part is related to the head, not to the neck. In the periphery of the spines there are the \emph{dynamic actin} filaments: these are twisted or attached in the center thanks to the actin polymerization proteins and on  one side they are attached to the surface of the cell membrane, via \emph{spectrin}.

Here we have to achieve 2 important features: anchor and dynamic possibility to elongate or restrict, There are proteins that bind the dynamic actin to the cytoskeleton in order to have more stable spines.

There is another molecule, the \emph{myosine}: actine and myosine contribute to the contractile of the cells. Here they have another important feature, the transport of vesicles and proteins and small organelles from the parental dendrite. 

When the spine is stable, the peripheral filaments are bound to the central core; when there is a need to elongate, the filaments can change length.

\subsection{Regulation of dendritic spines motility}
Cytochalasin D block the spines movement by blocking the actin polymerization. We can block polymerization by blocking Ca: there is no changing in the shape. But Ca is coming from intracellular stores or from outside? Both. 
At the end, we can block step by step the cascade and we can demonstrate that for having this phenomenon we need all the cascade.

\subsubsection{Cellular model for the memory}
we can have long-term potentiation (LTP) of synapses, but in some other cases memory is made of long-term depression (LTD) of the synaptic activity. One of the molecular an cellular mechanism to achieve this in to change the shape of spine: for LTP a bis head shape and so much actin filaments while for LTD is the contrary.
\\

The consequences of having different shapes is linked to LTP and LTD, but why?

There is a link with Ca: once the neurotransmitters arrive on the surface of the post-synaptic membrane and open the channels, there is a huge amount of Ca entering in the cell. It can go into the soma and make a signal there, or stay in the cytoplasm and activate signals there. Large spines have also a big smooth ER, that is the intracellular storage for Ca. So, there is Ca entering from the channels and releasing from the intracellular stores. thereis also a relationship between the dimention on the spine and the amount of CA: if it is small and isolated from the dendrite, we have a huge increase of CA, while when the spine is large, the concentration will be lower but also rapidly spread laterally in the dendritic shaft.

The rising in intraspines Ca can be through the channels and from the intracellular store: if we have 2 synapses close, we have a huge amount of Ca that enters in the stubby spine and the Ca stays inside the mushroom spine, so 1 synapse is very active and the other one is less active. If the Ca from the stubby one can enter in the mushroom one, the Ca can exit from the spine apparatus and this will make possible for this poorly active synapse (the synapse with the mushroom spine) to be stimulated equally well that the other one (the synapses with the stubby spine) (coordination).  So we have a syncronized activity of these synapses: if there isn't a syncronized activity, we don't have strong synapses or integration at the level f the soma.

\chapter{Lesson 6: Synapses}

What is the main function of a neuronal cell? To send messages. That's why we need a network! The shape of dendritic arborization is functional to the type of connection that we need.  We have an \emph{input layer} that collect the initial information, like the sensorial neurons located in the periferal ganglia; then a \emph{hidden layer}, which is the interneurons: then a \emph{output layer} that are the spinal motorneurors. Multiple interneuron segnals converge in a single output signal. This is the \emph{signals integration}.

Cajal and Golgi observed baskets of a very long and ramified axon and basket cells: at the end these cells have a boutons, like a claw(clava) that are covering the cell bodies of the neurons in the cerebellum. 

Sherrington understood that the impulse could be also an inhibitory signal, but he was wrong on the point that the transmission involve only electrical principles: Elliot was able to increase the heartbeat by incubating frogs with adrenaline (he was not believed). Few years later, Otto Lewi made a more convincing experiment (Elliot didn't start from the old models): a vagus nerve connected with a heart immersed in a physiological solution. He could stimulate electrically and  see contraction: but, if there is a chemical substance which is released ad the synaptic level, this substrance sould go into the physiological solution! In order to determine if this substance is sufficient to increase the beat, I could put the solution containing this compound with the heart not innvervated: he sow contraction and he was believed.

The presynaptic ending contains mitochondria and other organelles; the postsynaptic has neuroreceptor on the membrane; the space in between is called synaptic cleft, 20 nm, is also surrounded by astrocytes. The adult human brain contains between $10^{14}$ to $5 \cdot 10^{14}$ synapses. 

Synapses are funcitonal connection between neurons and between neurons and other cells.  Def: a presynaptic terminal that is able to synthesize a neurotransmitter, that is contained in vesicles.

\section{Criteria for Chemical transmission}
The most of synapses in the CNS and PNS are chemical synapses. There are some recurrent features of these synapses:

\begin{itemize}
\item{The neurotransmitter is synthesized in presynaptic terminals}
\item{Neurotransmitter is stored in secretory vesicles}
\item{The release is regulated}
\item{Presence of receptors in the postsynaptic membrane}
\item{The action of neurotransmitter is controlled by termination steps (like astrocytes that release ACh binding protein).}
\end{itemize}

\subsection{Classification of synapses}
The excitatory synapses are mainly located on spines while inhibitory on the dendritic shafts, soma and axons.The neurotransmitters for excitatory synapses are glutamate, ATP and ACh, the ones for inhibitory synapses are GABA and Gly. Those synapses contains channels, that create a pore in the plasma membrane: when the channel is open, ions can cross the membrane according to their potential. There are other type of synapses that have no channels: they are using a particular type of receptors, the \emph{metabitropic receptors}. In addition to ion channels, we can have receptor channels, line NMDA type for Glu: one part binds Glu and the rest of the subunit forms a ion channel. In metabotropic receptor, they do not form an ion pore.

Electrical synapses can work only if the membranes are very very close: like in GAP junctions. It works as a syncitium, even if structurally it is not because the cytoplasm is not shared.
 GAP junctions are not always open\todo{cosa le fa aprire e chiudere???????}.
 
The unidirectional synapse is a a-symmetry synapes: it is between an axon and a dendrite at the top of a spine and the membrane is thickening (\emph{postsynaptic density} )in just one side. 

Symmetric synapses are presents between axon and cell bodies usually.

The collaterals branching of the axon can have a synapse on each branch end. We can also have a single axons with \emph{boutons en passage}: everitime a axon touches a dendrite in a spine, it makes a synapse, so there are multiple synapses.

Typically, a-symmetric synapses are excitatory and symmetric synapses are inhibitory. The axon-somatic synapses is usually inhibitory.

So, synapses classification can be based on\todo{lo chiede all'esame!!!}:
\begin{itemize}
\item{cytoarchitecture}
\item{ method of signal conduction and}
\item{ conductance of the postsynaptic element}
\end{itemize}


\subsection{The synaptic bouton}
By freeze-hatching we can detach membranes and separate the lipid  bilayer to look at the protein that are inserted:  we can see what arrives at the surface of the pre-synaptic membrane. There is an array of trans-membrane proteins aligned: looking on other synapses we can see vesicles fused on the membrane, that occur laterally on the array of proteins. After the release of the content of these vesicles, the membrane is recovering and taking back with endocytosis part of the membrane, by coated-pits vesicles that has clatrin.

The vesicles are aligned along these arrays of transmembrane proteins, and they fuse laterally. 

\subsubsection{Neurotransmitter}
They can be aminoacids, monoamine (dopamine, noradrenaline, adrenaline, serotonine, melatonin), then ACh, adenosine, histamine, ATP etc. There are also neuroactive peptides: CGRP, SP, PY.

Monoamines are most regulatory neurotransmitters: they are often together in the same synapses.

The secretion of neurotransmitters is very very fast: all the vesicles are kept inside until the arrival of the action potential and the increase of Ca in the presynaptic terminal.

Ca controles:
\begin{itemize}
\item{Exocytosis}
\item{Mobilization of synaptic vesicles, that are linked to the cytoskeleton}
\end{itemize}

The consequence is the fact that, since these vesicles contain a defined amount of neurotransmitter, the synaptic transmission is quantal: a fixed amount of neurotransmitter cause a release oaf a certain amount of vesicles.  This is due to the anatomical specialization of the synapses and the properties of ion channels.

\chapter{Lesson 7: Organelles and secretion}
Synaptic transmission is a specialized type of exocytosis. We need to have a vesicle targeting and the release has to be event-specific and fast release of chemicals. The molecular biology reveals conserved components: 

ER \lfreccia AG \lfreccia vesicles to the bouton. Vesicles sometimes have to do a very long journey. The proteins inside the secretory vesicles are mainly \emph{enzymes} to produce the neurotransmitter, which has a precursor in the bouton: sometimes this process occur in the cytoplasm, sometimes in the synaptic vesicles. Vesicles proteins are involved in:
\begin{itemize}
\item{Trafficking}
\item{Exocytosys}
\item{Synthesis of neurotransmitter}
\end{itemize}

The cell must sort out the different vesicles produced by AG in a very specific manner. Neurons do have a ER and a AG in the cell soma, but they also have some peculiar characteristics: GA is huge and numerous, because neurons are highly metabolic active cells. Most proteins last 1 to 2 days, but some proteins that can last moths and proteins that can last few minutes, so that explains why there is a lot of ER and GA.

GA has many functions: modifies N-linked oligosaccharides and sorts proteins so that when they exit the trans Golgi network, they are delivered to the correct destination.


Vesicles with a coat made by   COPI are in the pathway from the GA to the ER; COPII from the ER to GA. In the vesicular transport models, vesicles brings back the enzymes to be recycled; in the cisternal maturation model there are some vesicles that move, basically backwards, to recycle enzymes. 

Clathrin, COPI and COPII are very different: clathrin forms the triskelion structure that imposes a change on the surface of the plasma membrane\footnote{Its shape is given by the cytoskeleton}. One of the key issues is that in order to bind to the membrane, clathrin needs receptors and adaptors: the ligands (the proteins that have to be secreated) binds to the recpetors and signals that the proteins are ready for being secreted \lfreccia the receptors recruites adaptine, that is attached outside the receptor \lfreccia signal to clathrin to bind \lfreccia formation of the secretory vesicles \lfreccia clathrine goes out after the formation of the vesicle. the detachment of the vesicles is made by dynamin, the squeeze the neck of the vesicles.

All these secretory vesicles are transported by microtubules: they are going from the ER to GA, and then motor proteins that can bind the vesicles.

A change in the pH from the Er to GA permits the detachment of the proteins from the receptors on the plasma membrane (that binds the ligand that has to be secreted). All coated vesicles contain GTP-binding proteins, very important for the assembly of the vesicle. These proteins are associated to different vesicles and different coated proteins.

The adaptor proteins provides the information on where the vesicle have to go, so the target\todo{see slides}. AP is a complex of different subunits. How a coated vesicle is created?

We need a receptor that binds to a ligand, that can be the secreted protein or the enzyme of the maturating cisterns. Trough the maturation, the cisterns reaches a different pH to activate different enzymes and increase (or decrease) the affinity for the receptor. The vesicles form only when the ligand is there: this prevent the formation of empty vesicles. 

In the regulatory secretion, we need a signal to start the secretion, while in the costitutive secretion aussi tot que the vesicle is created, it is fused with the membrane to secrete.

A secreted proteins has a signal peptide in the N-term that let the protein to go in the ER: this is sufficient to have the signal peptide in order to go to the secretory pathway; when a protein has to go on the regulated pathway, there is a signal sequence into the protein that causes a folding to bind a specific receptor. This give 2 information: the protein is a secretory regulated one and it's properly folded.

In the brain, the protein that stimulate the growth of neurons is a secretory protein, addressed both to the costitutive and regulated pathway, but for BDNF the 90\% has a regulatory pathway, the 10\% the costitutive one. In contrast, NGF (nerve grow factor), very similar in sequence and structures and receptor and pathway activated to BDNF, is secreted at 90\% with the costitutive pathway and 10\% with the regulated one. What makes the difference between these 2 similar factors?

There are 4 aa in the core proteins that when the protein is correctly folded form a signal for a receptor, the \emph{carboxy-peptidase E}: it recognizes the aa of BDNF. If we exchange the aa that differ from BDFN and NGF, the researcher saw that the pathways were exchanged. (in particular, the made 2 mutations. BDNF...).

So, how neurotransmitter are secreted?

Basically, the principle is the same: vesicle goes by a donor compartment to a target compartment (formation, transport and fusion). The neurotransmitter has to be stored in vesicles because:
\begin{itemize}
\item{They can undergo degradation, so they have to be protected}
\item{Allows for regulation}
\item{Provide a storage system}
\item{Can be docked at active zones}
\item{Differ for classical transmitter vs neuropeptides}
\end{itemize}

Synaptic vesicle components are made in the soma, in particular in the Golgi, and they are transported to their sites by the cytoskeleton etc. Once the vesicle is fused to the membrane, we have a endocytosis that forms and endocytic vesicle: two vesicle can fuse to forma endosome, that can go to the catabolitic route (late endosome) or form new vesicles. The late endosome, on one side produces new vesicles that come back to the Golgi, on the other side they become a lysosome. We also have the recycling of the neurotransmitter outside.

In neurons, we have he classical coated pits pathway or cisternae models, then the kiss and run model to recycling: the vesicles arrives to the membrane, that te vesicles (regulated pathway) are activated by the phenomenon of priming, then when Ca arrives they fuse (not completely( with the membrane. The neurotransmitter is secreted and the membrane closes back \lfreccia recycling of the vesicle without the clathrin pathway. If the vesicle fuses completely, we need energy to create the curvature to recycle it: this is given by clathrin (this is not the kiss and run model).: in this case we have a fusion with the endosome, an additional way to regulate because, in the case of kiss-and-run, most of the vesicle are filled by neurotransmitters and if we don't want to have transmission, the route of the endosome eliminates everything.


\chapter{Lesson 8: Cytoskeleton and axonal transport}

Synaptic vesicles and every vesicles that is transported at the axon terminus make a long journey along the cytoskeleton. 
 The function of the cytoskeleton are:
 \begin{itemize}
 \item{Dynamic  scaffold}
 \item{Internal framework to divide the different compartment of the cell}
 \item{Network of highways}
  \item{Provide the force to generate apparatus-cell movement}
 \item{mRNA anchoring}
 \item{Cell division}
\end{itemize}
 We have microtubules, microfilaments and intermediate filaments. MT and MF are made by a single protein; IF are made of several proteins.
 Nurofilaments, ore IF, are very heterogeneous. 
 
 These 3 types of filaments are differently distributed in the cell: the actin filaments are at the periphery of the cell, constituting the cortex, spines, microvilli etc. Actin can be elastic, so gives an internal coating and the shape, being on the periphery.  IF are inside the cytoplasm, attached into particular structures of the membrane, forming \emph{junctions}, anchoring 2 cells. MT irradiate from the centrioles.
 
 MT have a cavity inside, 13 colons of tubulin molecules ($\alpha$ and $\beta$ tubulin): they are resistant to compression and keeps the cell shape, in some cases are involved in cell motility, because they are shifting one on top of the other, making very large movements. They are involved in chromosome movements, organelle movement.
 
 MF are involved in the mantainance or change of the cell shape, in dynamic contraction.
 
 IF are a serious of fibrous proteins, rigid and super coiled: these proteins are of the keratin family. They have no polarity! Glial cells can be identified by staining these proteins of the IF, because they express a unique type of IF (type III). The type 4 proteins like NF triplet (heavy, medium and light) are important because they form tangles in many disease like Parkingson, Alzheimer.
 
MF have a polarization because they can be elongated by actin monomers only at one side. These monomers has to be bound to ATP to bind the filament. The de-phosphorylation of ATP into ADP make the dissociation of these monomers. The most difficult step is the \emph{nucleation}, then we have a \emph{dynamic stability} because the length of the filament won't change.

In MT, the polarity is given by the fact that we have a dimer. If it has at the tip only monomers bind to GDP (because of the loss of GTP cap), there is a catastrophic depolymerization. In addition to MT, there are accessory proteins, like MAP2 for the dendrites and tau for the axon: the distance by microtubules can be regulated by phosphorylation of these proteins. In excess of phosphorylation tau, there is formation of tangles in neurodegenerative diseases.

Studying that, the cell culture were so important: it provided the majority of the information.

In highly polarized cells like neurons, the presence of the cytoskeleton is very important: we need to have more space between the cytoskeleton filaments, to let the organelles flow along the axon, so the MT do not to be bound via covalent bonds. The spacing is very important and there are a lot of connections that create a very irregular an wide space of NF triplet.

These were the static properties of the cytoskeleton: now we consider the \emph{cell motility}.

\section{Cell motility}
Growth and retraction of cellular processes: the cell is emitting some filopodia to explore the environment and, when it finds a clue, there is a signal that goes in the cell and the advancement of the cell. We can have contractile bundles of AF because of the opposite organization of AF; in the central part of the cell, we have a gel-like network, very similar to the cortex, because there a re AF going in all directions, forming a mash; if we need to push a filopodium, we have the convergency of the AC with the + end polymerizing in one direction, at the tip of the exploring filament. When the filopodium finds an environment not suitable, the machinery of the actin stops the growth and the depolymerization occurs.

In filopodia we see parallel fibers. We can use electro-microscopy, but the immunofluorescence staining is more useful. Growth cones have very different shapes, the filopodia can be very long and highly motiles. The tubulin staining is in the axonal shafts , and this part is stick, while the part containing actin is flat, there are boundles and filopodia.

In the filopodium we have polarize F-actin boundles, with the + end at the tip. Then, a flat area in which we have the actin cytoskeleton forming  a network, and some \emph{exploratory microtubules} that follows the actin filament and, in case of establishment of this filopodium, they reinforce the structure to keep growing. In the axonal shaft we have a stable domain of MT; in the part of the actin domain we have some unstable MT. There are also some structures, the \emph{actin arc} that contribute to the central core of the growth cone. There are also some accessory binding proteins that contribute to the stabilization.

\subsection{Molecular details}

An extracellular stimulus activates GTPase ans PIP2: this activates a protein complex, the WASP, that binds to the GTP and is able to interact with Arp2/3 complex \lfreccia this is able to help enucleating the first actin filaments, that is the rate limiting step that requires energy (starting from the plasma membrane). Some ancillary proteins can make biforcation in the filaments (barbed ends). The \emph{cofilin}, an actin-depolymerizing factor promotes the dissociation of filaments, so give an highly dynamic form of depolymerization  and polymerization, by bringing the ATP-bind actin to the initial site.

\subsubsection{Role of microtubules}
MT have a distribution along the axon that is polarized (+ end to the terminal tip), due to the way in which they are assembled, and gives information about the transport. In dendrites, this is not always true: in the initial part we have a mixed polarity, while in the more distant part it is like in axons. 

Axonal transport is a very difficult task because in some cell axon can be more than 95\% of the neuronal volume. There are many components of the synapse that need to be transported: this was seen thanks to a ligature experiments, that showed an enlargement before the ligature, that disappeared at the removal of the ligature. 
There is a group of proteins that is moving along the axon with a fast axonal transport; a second group with an intermediate rate transport and a third group with a slow transport. Even the cytoskeletal proteins are transported because of their turn-over.

Are the MT transported along the axons already assembled or in monomers? This was solved by using tubulin made fluorescent by fusion with GFP: we load the neurone with that and see a fluorescent axon, than we flash with laser 1 point and bleach the fluorescence in a point to create a gap \lfreccia we see a stretch of MT that are moving in this gap! They are moving very fast and in both directions! Figoooooo.

The problem of the retrograde transport was demonstrated in this way: ligation of the nerve + fluorescence. In the initial situation, we have molecules which are labeled, in red anterograde moving and in blue retrograde moving: after the ligation we have on one side an accumulation of red proteins and on the other side an accumulation of the blue proteins.

Another experiment can be made by labelling the motor proteins: the ones that are involved in the retrograde transport (dyneins) are on both sides of the ligature:that is because they are produced in the soma, so they are first transported on the terminal and than back to the soma.


\chapter{Lesson 9: Cellular transport and molecular motor}
There are two types of microtubule proteins: kinesins and dyneins. Another family f proteins uses actin for trafficking, this is the myosin superfamily. They have separated motor proteins, because the transport from the center via microtubuels end is in the terminal, so the coming back is helped by actin filaments.

Also cilia and flagella use one of such motor proteins, in particular the dyneins. 

Myosin II is a muscle motor proteins: it has a coil-coiled of 2 $\alpha$ helices and 2 globular heads linked by a neck to the coil-coiled. The globular structures are useful to interaction; in the part between the heads and the filament, there is a sort of light chains that allow the changing in conformation. When they are assembled together with actin and look like a millepiedi.

Even the kinesins have a similar structure. The motor domain changes in the class, so we can classify the kinesins ad N-kinesins (N-term), C-kinesins (C-term) and M kinases (middle). These kinesins proteins works as a monomer or a homodimer. 

C-kinesins and dyneins are minus-end directed; N-kinesins (ressemble the myosin) are plus-end directed; M-kinesins are involved in the depolymerization of the microtubules at the + end.

Depending of the binding of the heads to the $\alpha$ or $\beta$ tubulin, they go in different directions; in N-kinesins, the head is directed in the - end, the "tails" in the + end, so the kinesins can move only in the - end. These proteins are capable to transport huge vesicles and organelles so it is probable that these molecules act in groups to transport them.

General features about motor proteins:
\begin{itemize}
\item{They generally mone unidirectionally}
\item{They move stepwise: this is given by the structure and the fact that they occur as a dimer}
\item{Series of conformational changes: due to chemical cycle and mechanical cycle that requires energy. ATP binds to the motor, than hydrolysis, that release of ADP and Pi and then binding of a new ATP.}
\end{itemize}


The hydrolysis causes the attach and the release of ADP causes the movement and the detachment. 


\section{In the neuron}
MT are generated in the centrioles, they elongate in all directions and the + end are in the periphery, where we have the growth of MT: in axons is the same, and while in the cell body MT are directly linked to the centrioles, here we can have some discontinuities. Kinesins motors goes from the cell soma to the cell periphery, while dyneins go from the periphery to the soma.

In transport, there are families and other proteins that allows to discriminate the cargo: 
\begin{itemize}
\item{Synaptic vesicles: they have to go only in axons}
\item{MT: they have to go  everywhere in the cell}
\end{itemize}

There are kinesins that transport from the trans-golgi network to dendrites, like KIF7 and KIF5, they transport \emph{dense core}, vesicles with proteins\footnote{synaptic vesicles have a clear core} like membrane receptors. The most important receptors in dendrites are the receptor for neurotransmitter: having them at the proper place, in the right amount ecc is very important. In the axons, we can also transport mitochondria, but the transport could take hours or days: there are some specific transporters that transport the cytoskeleton, to contribute to the formation of axons collateral. 


\subsection{Kinesins}
They are involved in ATP hydrolysis and binding to the MT (in the globuler head). There are light chains that bind the vesicles. They share a molecular domain, then there is the coil-coiled domain and additional domains that "sense" the environment, to target the right cargo to the right place. 
 
THe structure of the molecule has been seen with electronmicroscopy. The classification was a mixture between genetical analyses and morphological analyses. 

There are different kinesines that can transport the same organelle: KIF3 and KIF17 are founf in dendrites and transport different cargos because they have different adaptors, like GRIP. An additional group of proteins involved in transport is KLC: they are involved in the axons and they aren't present in the dendrites.

GRIP1 is related to the ampa type of the Glu receptor; KIF5 can use different adaptors, like milton and miro for mitochondria. Considering the synaptic vesicles, there are some cargos that contains only them and other cargos that contains only the protein for the machinery of the docking site. 

The principle of inhibiting the usage of the material during the journey is to avoid secretion and loss of the components. 

The different Glu receptors use different vesicles. 

The signals that these motors can send are several: there is one class of cargos regulated by Calmodulin-kinases. When Ca high, the cargo in detached; that is useful to deliver something in a region rich in Ca, like a synapses! By this mechanism we can deliver the GluR in post-synaptic membrane, like in spines. Another mechanism is sensitive to GTP, in particular to Rab proteins, that are involved in reacting to protein kinases that active the death domain. In this place we can have energy by GTP: a synaptic vesicle precursor contain the enzyme to produce the neurotransmitter.

Another mechanism has the miron milton proteins, that attach a small kinesins and remains attached to the mitochondria.

\subsection{Cytoplasmic dyneins}
There a re multiple subunits: they have 2 feet that can move, but the foot is not a simple globular domain like in kinesins, because there are multiple subunits, so they can be much more regulated that kinesines. This protein complex can accept multiple cargos.

We have intermediate chains, that are dimers: 2 genes, it can be Pi. Ligth intermediate chains also can be Pi, are a dimer and controlled by 2 genes. The heavy chains are also dimers.

In time laps analysis, there are spots moving: they are the cargos. We can plate and bound the motors to a glass surface, and then put the MT, giving ATP: plating kinesines, the MT structure moves in a direction in which the barbed end in in the tail; plating the dyneins, the stretch is moving with the barbed end in front

If we make a ligature, we see that kinases accumulate before the ligation; dyneins are in both sides! That's logical, because both these motors have to be produced in the soma, so also dynein is transported to the plus end \lfreccia they are transported in aggregate not to be functional until they reach the + end of the MT. When label in red kinases and in green dyneins, in the anterograde transport are co-localized, but when they move retrogradely they are not. So, the revised model is the following: the trick is to have the kinesine molecule transporting a non-functional dynein structure on the top. In case of dyneins, the steps can be long or short: this depend on the ATP loaded.


Kinesines can be transported back like dyneins are transported to the + end, but it is not clear what occur from the detachment of the cagro from the kinesin and the retrograde transport of kinesin.

In dyneins, the entire structure is moving: the head is composed of ATPases and when ATP is at low level, we have very long step (25 nm); with intermediate load, 15 nm and with a high concentration ATP is 8 nm, probably some quick steps. 



  












\end{document}
