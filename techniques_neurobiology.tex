\documentclass[a4paper, 12pt]{book}
\usepackage[english]{babel}
\usepackage[T1]{fontenc}
\usepackage[utf8]{inputenc}
\usepackage{todonotes}
\usepackage{caption}
\usepackage{subfig}
\usepackage{textcomp}
\usepackage{float}
\captionsetup{format=hang,labelfont={sf,bf}}
\usepackage{amsmath, amssymb}
\usepackage{xspace}
\usepackage[hidelinks]{hyperref}
\usepackage{mhchem}


\newcommand{\freccia}{\ensuremath{\rightarrow}}
\newcommand{\lfreccia}{\ensuremath{\longrightarrow}}
\newcommand{\gradi}{\textcelsius\xspace}
\newcommand{\ang}{\AA \xspace}



\begin{document}
\author{Elisa Nerli}
\title{Techniques in Neurobiology}
\maketitle
\newpage
\tableofcontents
\newpage


\chapter{Lesson 1}
Gabriele Baj gbaj@units.it
\section{Scientific method}

\begin{itemize}
\item{Purpose}
\item{Research}
\item{Hypothesis}
\item{Experiment}
\item{Analysis: can lead to a confirm of the hypothesis or a rejection}
\item{Conclusion}
\end{itemize}

Purpose: how it works? We need a characterization at \emph{molecular} level of the problem. One of the main purpose is to find a pathology cure, to find a therapy. The target identification allows to think to what is wrong. 

Book: research techniques in Neuroscience.

Lab: 14-15.30, 15.30-17, 17.30-19.
\section{Models in Neurobiology}
We can use humans (like for alzheimer disease), whole brain, organotypic slices or cultures, reaggregate cultures or organoids, dissociated primary cultures and cell lines (from high level of complexity to a lower).
\begin{itemize}
\item{Cell lines: human or animal cells, generally coming from organs, cancer cells or immortalize cells, in which the right gene placed in the right position led these cells to replicate an survive. The number of chromosomes of these cells may be different. This is the less predictable as a model. HELA cells useful for neuroscience.}
\item{Primary culture} 
\item{iPScs derived neurons/glia (induced pluripotent stem cells)}
\item{Organotypic cultures}
\item{Animal model: they are not cheaper!!!!!!!!!!}
\end{itemize}

Cultured slices must be thin. 

Primary tissue culture: a culture derived directly from a tissue. The good things is that in 2D they really resemble the natural tissue, but they have limited growth potential (none for the neurons).The immortalization of these cells is very popular: with elena cattaneo they made an immortalization of these cells for huntington's disease, to study the biological function of the hungtintin protein.

\subsection{Primary neuronal cells in culture} 
Spinal cord, cortex don't generally have nSC, but they can produce primary culture, especially from the age of the animal, like P0 or P1; olfactory bulb, cerebellum and hippocampus have neuronal stem cells nSC. Youngest is the tissue, higher is the number of cells that will survive to the procedure, because young cells respond better to manipulation and treatments. A P20 rat is already old, but useful for glial cells. Angelo Vescovi took neurons from a natural aborted fetus of 3-4 months for Parkinson disease. They will use ESc for this, but remeber that iPScs injected in an animal risk to produce a cancer (they are theratogenic). Someone is producing stem cells mutated with inside some genes that will kill these cells when becoming a cancer.

Explant culture: cut the head, extract the brain, take some fragments and dissociate cells by enzymes and then plate cells. In some media there are some glial cells next to the neurons, to recreate the milieu of the neurons.

From a brain, we can decide to have a primary culture or an organotypic culture: if you want to study something related to the development, you need a primary culture; in organotypic culture cells cannot communicate, do not receive signals from outside and then dies after few days, but the hyppocampi part is different because there are 3 synaptic loops\todo{guarda il film memento} and all the signals enter into the perforant path until the CA1: these loops made the tissue think there are always signals.

(19-10-2015)

\subsection{Cell strain (extended culture)}
 Now they are called immortalized cells.  Expertise is not needed, these cells have perfectly controlled environment and controlled and defined physiological conditions.The cells are particularly homogeneous, so they are clone of one cell, while in a primary cell culture we have different cells. These cells are economical and "fast", they need less reagents than in vitro. Cells in primary culture are so slow! One of the possible disadvantages is that these cells are unstable on the point of view of chromosome number: often these cells have 70 or more chromosomes. 
\\

Primary cells are taken from a tissue, so they don't come from a tumor, have a certain ability to replicate and then they will stop. In cell lines, the cells can replicate more, but they don't generate a tumor, they just loose the diploid/euploid condition. 

Primary cells are normal, cell lines are transformed (usually); both are considered as non-tumorogenic: this is not true for IPs or stem cells, and partially true for cell lines: cell lines are not tumorogenic in an organism with a perfect immunity system. Primary cells have a density limitation of growth because of contact inhibition: they understand that there is a cell close and, for that reason, the density of growth is limited. Cell lines generally likes to stay attached, the even kill each other or grow intrecciate. For that reason, primary cells are considered a monolayer while cell lines not.  

In cell lines it is possible to induce the steady-state and syncronise the culture to a certain phase of the cellular cycle. They require low serum while primary cells requires a lot, because they produce only 3-4 aa.

Primary calls cannot stay alone, because they share some reagents and some messengers. Cell lines do not express some specific markers, but they can be identified looking at the cell chromosome number.

Primary cells retains their specific functions, while in cell lines is often lost: each one of them retains some functions, that's why we use more than 1 cell line. All these details are related to the limitation of the techniques that we need to use in cell lines or primary culture\todo{ATCC is a cell bank, a website where you can find information about the cells that we will use in lab}.

The coating for primary cell may be different: we need to make a choice because every kind of coating will give us a different result.


\subsection{Organotypic cell culture}
There is a chopper that cut the tissue in very thin slices: the ippocampi is generalli 400 $\mu m$. That0s the better way to study alzheimer disease or Parkinson: we can put the proteins on the slices to see how they are toxic like $\beta$-amyloid. Right now, the main directon in  research in Alzheimer is no more the $\beta$-amyloid.

This cultur allow to study the three synaptic loop in hippocampi: generally, 90\% on neurons put in culture become glutamine-ergic or GABA-ergic. The organotypic slices allows to get near to animal model. It's like to have a "window" on the tissue, so we ca give the drug with a fluorescent molecule, something that we cannot do in vivo.

This is considered in vitro culture: thy are derived from explants of undifferentiated embryoni brain, spinal cord, sensory organs. 

The advantages are the 3D organization and they can survive in culture for a week to a month, depending on the culture conditions. For a fast read-out, use cell lines; for a pretty fast read-out with neurons, use primary culture; for visualizing on a 3D image the aging, use the organotypic slices.; for animals, it takes years.

The disadvantages are the difficulty in quantifying small changes in differentiation or viability, depending on the parameters measured. The nutrient supply is not so fully manageable, like in primary cells. Often, in prolonged time, there is an hypoxia of the slice. Also, only a small piece of the tissue retains the in vivo-like physiological conditions.

Generally, we can do many things with organotypic cell culture, like electrophysiological studies. 

\section{Techniques and general read-out}
In vivo:
\begin{itemize}
\item{Behavioral}
\item{Metabolism}
\item{Toxicology}
\item{Electrophysiology}
\end{itemize}

When we are in vivo, we work only considering the \emph{known target}: in all research we can have different target, like DNA, RNA, proteins or other elements, like heavy ions, and interaction between these targets, like DNA-protein, RNA-protein, protein-protein etc. All of these target can be considered unknown or known: unknown is the so called "not-hypotesis driven research", like the DNA microarray method. Then we have the known target, when we want to identify a protein or DNA or RNA.

\subsection{Unknown targets}
Microarrays for mRNA, miRNA, ncRNA, SNP or protein \lfreccia biochemical read out.

Also the bioinformatic approach is useful: it gets a biochemical read out.

\subsubsection{cDNA Microarray}
The target is to identify which mRNA is different between a pathology and another condition: extract the mRNA, produce a cDNA with reverse transcription, label one strand of cDNA and hybridize on the cDNA microarray. On the microarray we have the antisense of all the possible mRNA present human. 

\subsubsection{2D-gel electrophoresis}
Run for the weight and the pH. 

\chapter{Microscopy}
(29-10-2015)

We need so many techniques because we have to see different things. Light has a huge dimension, especially in biology: the light that we see 400-700 nm, so we cannot go under a certain dimension, that is the \emph{optical resolution}. The physical limit of resolution is half of the wavelength we use. 

In the regular microscope, the light coming from the object is elongated by the objective, taken the the focal plane and then sento to the eye. Now most of the microscope has an objective that can produce a parallel ray of light, thanks to another lents, the tube lens. 

We're not seeing the object, but its image projected on our retina. We experience the optical alteration. The microscope lens change the relative dimensions, change shape and colors.
All of the evolutions in microscopy is to reduce aberration: light is a sum of different wavelength, so the lens don't act as a prism in the center, but they show a diffracted light at the borders. The diffraction produces all the aberration in color: this diffracted light is cause by the diffraction pattern of each material. This can be reduced using oil, so the light coming out from the glass thinks that is still in the glass (same numerical aperture, that is the ability of a microscope to gather light and resolve fine specimen detail at a fixed object distance)\todo{www.olympusmicro.com oppure nikon microscopyU}. A high NA has small working distance, we have to stay near the sample.

\paragraph{Resolution of microscope}: the single poit ol light illuminated gives a graph like the PSF, the point spread function. The light create a airy disk pattern, and the dimension is determined from the optical system and the light used. Smaller these disks are, higher is the resolution. The PSF determine the resolution of the microscope, so when we have 2 points in the sample. like 2 nuclei so near that the light arrives sovrapposta, we cannot understand if the 2 point are separated or not, so the "resel" (misure of the disk) is the measure of the resolution.

In many case we use UV light to excite, but this don't pass the glass.

Higher is the magnification, lower is the ability to have focus on different plans of the image.

Bright-field microscope: WE SEE THE LIGHT blocked by our sample (except for the light of the color). We can use several lenses and a total magnification. If the slides is transparent, we won't se anything.

Dark-field microscope: made with DIC, we block with polarizers (one before the sample and one after, 90 degree placed) the light. Te sample change the polarization of the light and a part of the light skip a part of the cage: the final effect is to have a bright object on a dark background. The sample must me bi-rifrangent. 

Phase contrast microscope

Fluorescence-microscope: we want to see a single element. Fluorescence light is a no more polychromatic light. Some of the molecules in the sample are fluorescent, but we have to illuminate the sample through the objective! The light is lost, so we collect the fluorescence emitted by the sample.
Phalloidin is a toxin that blocks the cytoskeleton, and it is possible to make a Actin-stain with phalloidin conjugated with fluorofores. With the fluorescence, we can eliminate the elements that we are noti interested in. A dichroic filter is a mirror for a wavelength, but lets pass the light produced from our fluorescent molecule.

All molecules have an absorption and an emission spectrum.

\paragraph{Confocal microscope}
The peculiarity is the \emph{pin hole}: that's for make the light reaching only the objects on focus. The light must be coherent and hits the sample like a "clessidra", the sample will produce information just from that part. We divide the sample in optical plans, and only 1 plan is on focus, so the light of the objective has the same focality in that plan.














\end{document}